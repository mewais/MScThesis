%% The following is a directive for TeXShop to indicate the main file
%%!TEX root = diss.tex

\chapter{Abstract}
Caches are essential to today's microprocessors. They close the huge speed gap between processors and memories. A bigger cache size should decrease misses and allow processors to perform better, but it's also more silicon expensive. Instead of increasing cache size, effective cache capacity can be substantially increased using cache compression. Recent proposals for previous cache compression techniques have been focusing on just one granularity, either compressing inside one line, or compressing similar lines together. In this work, we combine both compression techniques to leverage both inter-line and intra-line compression. We find that combining both techniques can result in better compression than previously described methods. On the other hand, it also complicates the cache structure and presents some implementation that we discuss.

% Consider placing version information if you circulate multiple drafts
%\vfill
%\begin{center}
%\begin{sf}
%\fbox{Revision: \today}
%\end{sf}
%\end{center}
