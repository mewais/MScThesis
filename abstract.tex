%% The following is a directive for TeXShop to indicate the main file
%%!TEX root = diss.tex

\chapter{Abstract}
Caches are essential to today's microprocessors. They close the huge speed gap between processors and memories. However, cache design presents a design tradeoff. A bigger cache size should increase performance and allow processors to perform faster, but it is also limited by its silicon, area, and power consumption costs. Instead of increasing the cache size, effective cache capacity can be substantially increased if the data inside the cache is compressed. Current cache compression techniques focus only on just one granularity, either compressing inside one line, or compressing similar lines together. In this work, we combine both compression techniques to leverage both inter-line and intra-line compression. We find that combining both techniques can result in better compression than previously described methods. We study and address the design considerations and tradeoffs that arise from such design and present an implementation that achieves the best possible compression and performance while maintaining overheads as low as possible.

% Consider placing version information if you circulate multiple drafts
%\vfill
%\begin{center}
%\begin{sf}
%\fbox{Revision: \today}
%\end{sf}
%\end{center}
