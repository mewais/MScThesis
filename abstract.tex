%% The following is a directive for TeXShop to indicate the main file
%%!TEX root = diss.tex

\chapter{Abstract}
Caches are essential to today's microprocessors. They close the huge speed gap between processors and memories. However, cache design presents an important tradeoff. A bigger cache size should increase performance and allow processors to perform faster, but it is also limited by its silicon, area, and power consumption costs. Today's caches often use half of the silicon area in processor chips and consume a lot of power. Instead of physically increasing the cache size, effective cache capacity can be substantially increased if the data inside the cache is compressed.\par
Current cache compression techniques focus only on one granularity, either compressing inside one cache line, or compressing similar cache lines together. In this work, we combine both compression techniques to leverage both inter-line and intra-line compression. We find that combining both techniques results in better compression than previously described methods, and also maintains the same performance as a normal uncompressed cache when running incompressible applications. We study and address the design considerations and tradeoffs that arise from such design. We address issues related to the design like cache structure and replacement policies. Then we present an implementation that achieves the best possible compression and performance while maintaining overheads as low as possible.

% Consider placing version information if you circulate multiple drafts
%\vfill
%\begin{center}
%\begin{sf}
%\fbox{Revision: \today}
%\end{sf}
%\end{center}
