%% The following is a directive for TeXShop to indicate the main file
%%!TEX root = diss.tex

\chapter{Conclusions and Future Work}
\label{ch:Conclusions}
\section{Conclusions}
Cache compression allows caches to perform as if they are of bigger size. It allows the cache to store more data rather than increase its physical size. There are multiple proposals for different cache compression techniques, but they all focus on just one granularity of compression, missing the opportunity of other compression granularities.\par
In this work we proposed combining two different compression granularities, intra-line cache compression in the form of Base Delta Immediate (BDI) compression, and inter-line compression in the form of deduplication. The result was a compressed cache that is slightly bigger and slower than its conventional counterpart but also faster and allows for more performance. Our cache allows for up to 2.25X speedup over a conventional cache, more than a bigger conventional cache of twice the size could achieve, while maintaining the area overhead around 20\% and the power overhead 33\%.\par
\section{Future Work}
Future additions or enhancements of this work include the following:
\begin{enumerate}
    \item Researching ways to decrease the overhead of using DedupBDI. For example, it's unlikely that all data lines in the cache will be compressed to just one segment, so the overhead of metadata per segment might be unnecessary and can be reduced.
    \item Investigating better replacement policies for the hash array. Hash arrays are critical for finding deduplication candidates. Allowing them to keep the useful hashes will positively affect the overall compression.
    \item Investigating the possibility of extending this work to support dictionary compression algorithms. CPACK~\cite{cpack} and DISH~\cite{dish} have a one line and a four line granularity, respectively. Extending this granularity to be cache wide might be useful.
\end{enumerate}